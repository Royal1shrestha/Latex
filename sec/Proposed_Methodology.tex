\newpage

\section{PROPOSED METHODOLOGY}
\subsection{Software Development Life Cycle}
For the project, spiral model was chosen among all others as it provides a systematic and iterative approach to software development. It is based on the idea of a spiral, with each iteration of the spiral representing a complete software development cycle, from requirements gathering and analysis to design, implementation, testing, and maintenance. Spiral model is one of the most important Software Development Life Cycle models, which provides support for Risk Handling. A software project repeatedly passes through these phases in iterations (called Spirals in this model) as shown in the figure below. The Radius of the spiral at any point represents the expenses (cost) of the project so far, and the angular dimension represents the progress made so far in the current phase. In each phase
of the Spiral Model, the features of the product dated and analyzed, and the risks at that point in time are identified and are resolved through prototyping.

\begin{figure}[h]
    \centering
    \includegraphics[width=0.9\textwidth]{spiral_model.png}
    \caption{Spiral model}
    \label{Figure 1: Spiral Model}
\end{figure}

\begin{itemize}
    \item Planning Phase:
    Requirements are gathered from the customers and the objectives are
identified, elaborated, and analyzed at the start of every phase. Then
alternative solutions possible for the phase are proposed in this quadrant.

    \item Risk Analysis Phase:
    During the second quadrant, all the possible solutions are evaluated to
select the best possible solution. Then the risks associated with that
solution are identified and the risks are resolved using the best possible
strategy. At the end of this quadrant, the Prototype is built for the best
possible solution.

    \item Development and Testing Phase:
    During the third quadrant, the identified features are developed and
verified through testing. At the end of the third quadrant, the next version
of the software is available.

    \item Evaluation and Planning of next Phase:
    In the fourth quadrant, the Customers evaluate the so far developed version of the software. In the end, if the Customers are satisfied with the software, then, it is released, else, planning for the next phase is started.

\end{itemize}

\newpage
\subsection{System Development Tools}
\subsubsection{Flutter}
Flutter apps are written in the Dart language and make use of many of the language's more advanced features. Flutter's engine, written primarily
in C++, provides low-level rendering support using Google's Skia graphics library. \cite{Flutter}Additionally, it interfaces with platform-specific SDKs such as those provided by Android and iOS. It implements Flutter's core libraries, including animation and graphics, file and network I/O, accessibility support, plugin architecture, and a Dart runtime and compile toolchain.

\subsubsection{React JS}
ReactJS is a declarative, efficient, and flexible JavaScript library for building reusable user inferface. \cite{ReactJS} It is an open-source, component-based front end library which is responsible only for the view layer of the application. The main objective of ReactJS is to develop User Interfaces (UI) that improves the speed of the apps. It uses virtual DOM (JavaScript object), which improves the performance of the app. The JavaScript virtual DOM is faster than the regular DOM.

\subsubsection{Node JS}
Node.js is an open-source, cross-platform, server-side JavaScript runtime environment. It makes simple for developers to build scalable network applications. It is effective and lightweight because it uses an event-driven, non-blocking I/O architecture. We can save time and effort by using its large library of pre-built modules into our projects. Web servers are frequently created with Node.js.

\subsubsection{Express}
Express is a minimal and flexible Node.js web application framework that provides a robust set of features for web and mobile applications. It provides a thin layer of fundamental web application features, without obscuring Node.js features that you know and love. Many popular frameworks are based on Express along with With a myriad of HTTP utility methods and middleware at your disposal, creating a robust API is quick and easy.

\subsubsection{Figma}
Figma is a cloud-based design and prototyping tool that allows designers and
developers to create interactive designs and collaborate in real time.\cite{UI&UX} It is a versatile tool that is used to create UI/UX designs, prototypes, and even graphic design projects. Figma offers a range of features like vector editing tools, customizable design components, and design libraries that make the design process more efficient and effective. It also allows all the team members to work together in real-time and removes the requirement of file sharing among members.

\subsubsection{MongoDB Compass}
MongoDB Compass is a graphical user interface (GUI) tool for MongoDB which is a popular document-oriented NoSQL database. It provides database administrators with a visual representation of their data, making it easier to manage, analyze and optimize their MongoDB databases. It offers users the chance to explore and manipulate data using an intuitive interface, providing a range of features such as data visualization, data filtering, and schema exploration.

\subsubsection{Git/Github}
\cite{Git/Github}Git is a free and open-source distributed version control system designed to handle everything from small to very large projects with speed and efficiency.Git is software that runs locally. Your files and their history are stored on your computer. You can also use online hosts (such as GitHub) to store a copy of the files and their revision history.Git can automatically merge the changes, so two people can even work on different parts of the same file and later merge those changes without losing each other’s work.

