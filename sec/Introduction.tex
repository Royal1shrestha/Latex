\newpage
\pagenumbering{arabic}

\section{INTRODUCTION}
\subsection{Background}
As the global shift towards sustainable transportation gains momentum, the adoption of electric vehicles (EVs) stands as a promising solution to reduce carbon emissions and mitigate environmental impact. In the context of Nepal, a country prone to the effects of climate change and environmental degradation, the integration of EVs presents a unique opportunity for a greener future.

In this project, a simple user-friendly mobile application is developed through which we can easily find and book nearby available charging stations with required specification along with additional features such as fuel station finder and repair garage finder. As of now, there are 60 DC fast chargers and 300 AC chargers available across the country.

\subsection{Motivation}
The motivation behind this project is to  facilitates the widespread adoption of EVs and to mitigate the anxiety among EVs owners and potential buyers about the lack of knowledge of charging station and garages available for EVs as well as conventional car. The app directly addresses the accessibility concerns faced by both electric and CV users across diverse terrains.

\subsection{Problem Statement}
Nepal's transportation landscape grapples with two main challenges: inadequate charging infrastructure for EVs and limited access to real-time fuel information for CVs. \cite{Mehrjerdi2019Electric}EV users often face "range anxiety" due to the scarcity of charging stations, while CV users lack a centralized platform for locating fuel stations and comparing prices. Similarly, there are some BMS problem in two wheelers as there is limited range of about 70 km. This project aims to bridge these gaps by developing an innovative app that addresses both EVs and CV users' needs.

\newpage
\subsection{Objectives of Project}
\begin{itemize}
    \item To provide up-to-date comprehensive information on charging station.
    \item To navigate fuel station, repair garage as well test drive center.
\end{itemize}

\subsection{Scope of Project}
The proposed proposal can help in route planning by incorporating charging and fuel station information into navigation systems. It  helps to eliminate range anxiety and ensures a smooth and uninterrupted travel experience. It contain repair garage finder helping in separating repair center for EVs and CVs. The test drive application can help in suggesting test drive for new EVs as well as EVs according to your specifications, cost, range and brand.

\subsection{Potential Application}
The proposed proposal holds significant potential for various stakeholders in Nepal's transportation ecosystem. EV users will benefit from reduced range anxiety and seamless trip planning, while CV users can make informed fueling choices based on real-time data. The app can serve as a catalyst for increasing EV adoption rates, supporting the government's sustainability goals. Moreover, the technological framework developed for this project could be extended to other regions facing similar transportation challenges.

\newpage
\subsection{Originality of Project}
While there are existing EV charging station locator apps, the integration of a fuel station finder as well as repair garage finder feature within the same app is a novel approach that caters to a wider user base and addresses the needs of both EV and CV owners. This unique combination enhances user convenience and promotes sustainable mobility while distinguishing the project from conventional standalone charging station locator apps.

\subsection{Organization of Project Proposal}
The content of this report is organized into 8 chapters. The first chapter is the introduction where brief details of the project like what this project is trying to achieve, its objectives, scope, and application are described. The second chapter contains a brief
description of the source materials and APIs we studied for this project. The third chapter is the analysis of the requirements, detailing various software, libraries used and operational feasibility. The fourth chapter explains about the architecture of our system, how the system is built, how its various components are connected and how they function together as a system. The fifth chapter is about the implementation details, enumerating how each individual component is implemented in the system. This
chapter describes how the components communicate with each other, and what measures were applied for the efficient working of the system. The sixth chapter is dedicated to the results achieved in the project. The seventh chapter focuses on the possible enhancements that can be done in the future. The eighth chapter will have the conclusion of the project, entailing our success and hindrances while completing the project. Finally, the appendix will contain the timeline of the development of the project.